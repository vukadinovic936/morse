%%%%%%%%%%%%%%%%%%%%%%%%%%%%%%%%%%%%%%%%%%%%%%%%%%%%%%%%%%%%
% % Title and the authors % %
%%%%%%%%%%%%%%%%%%%%%%%%%%%%%%%%%%%%%%%%%%%%%%%%%%%%%%%%%%%%
\title{Morse Theory}
\author{ Miloš Vukadinović \\ Veronika Starodub \\ Nikolay Ninov} \date{\today } 

\documentclass[]{article}
\usepackage{graphicx}

\usepackage[utf8]{inputenc}
\usepackage{setspace} \usepackage{amsthm} \usepackage{amssymb}
\usepackage{amsmath}
\usepackage{tikz}
\usetikzlibrary{matrix}

\newtheorem{theorem}{Theorem}
\newtheorem{lemma}{Lemma}
\newtheorem{proposition}{Proposition}
\newtheorem{scolium}{Scolium} 
\newtheorem{definition}{Definition}

\newcommand{\R}{\mathbb{R}}
\newcommand{\Z}{\mathbb{Z}}
\newcommand{\imply}{\Rightarrow}

\graphicspath{{images/}} 

% \doublespacing
\begin{document} 
\maketitle

\begin{figure}[h]
    \centering
    \includegraphics[width=0.50\textwidth]{cover}
\end{figure}
 
\tableofcontents
\section{Introduction}
%%%%%%%%%%%%%%%%%%%%%%%%%%%%%%%%%%%%%%%%%%%%%%%%%%%%%%%%%%%%%%%%%%%%%%%%%%%%%%%%%%%%%
% % begin N1
    \paragraph{\textbf{Morse Lemma.}}
        Suppose that the point $a \in \mathbb{R}^k$ is a nondegenerate
        critical point of the function $f$, and
        $$
            (h_{ij})= \left(  \frac{\partial^2f}{\partial x_i \partial x_j} (a) \right)
        $$
        is the Hessian of $f$ at $a$. Then there exists a local coordinate system
        $(x_1,\ldots,x_k)$ around $a$ such that
        $$
            f=f(a)+\sum{h_{ij}x_ix_j}
        $$
        near $a$.
        \\
        or
        $$
            f=-x_1^2-x_2^2-\ldots-x_\lambda^2+x_\lambda^2+\ldots+x_n^2+f(a)
        $$
        where $\lambda$ is the index of $f$ at $a$.

    \paragraph{\textbf{Theorem 1.1}} \textit{ 
        Let $f$ be a smooth function in a neighborhood 
    $N_x$ of $x=(x_1,\ldots,x_n)$ in $\mathbb{R}^n$. Suppose $f(0,\ldots,0)=0$. Then, 
    there exist $n$ smooth functions $g_i,\ldots,g_n$ defined on $N_x$ such that $g_i(0,\ldots,0)=\frac{\partial f}{\partial x_i}(0,\ldots,0)$
    for every $i$, and
    }

    $$
        f(x_i,\ldots,x_n)=\sum_{i=1}^{n}{(x_i,\ldots,x_n)}
    $$

    \paragraph{\textbf{Theorem 1.2}} \textit{ (Inverse Function Theorem)
        Let $f:\mathbb{R}^n\to \mathbb{R}^n$ be a smooth function on an open set $U$
        containing $a\in\mathbb{R}^n$. Suppose that $\det J_f(a)\not = 0$.
        \\Then there is an open set $V \subset \mathbb{R}^n$ containing $a$ and an open
        set $W\subset \mathbb{R}^n$ containing $f(a)$ such that $f:V\to W$ is a 
        diffeomorphism.
    }

    \paragraph{\textbf{Proof.}} Let $p_0$ be a nondegenerate critical point of the function
     $f:\ M \to \mathbb{R}$, where $M$ is an $n$-manifold. The degeneracy of the point
     $p_0$ on $f$ is determined independent of our choice of a local coordinate system.
     Therefore, we may assume that when we pick a local coordinate system $(x_1,\ldots,x_n)$ defined
     in a neighborhood $N_{p_0}$,

     \begin{equation}
        \frac{\partial^2f}{\partial x_1^2}(p_0)\not = 0\tag{1.1}
     \end{equation}

    or that we may pick a suitable linear transformation of the local coordinate
    system such that equation $1.1$ is true. We may further assume that $p_0$ corresponds
    to the origin $(0,\ldots,0)\in \mathbb{R}^n$ on the local coordinate system and that $f(p_0)=0$,
    replacing $f$ with $f-f(p_0)$ if necessary.
    
    By Theorem 1.1, there exit $n$ smooth functions $g_i,\ldots,g_n$ defined on
    $N_{p_0}$ such that

     \begin{equation}
        g_i(0,\ldots,0)=\frac{\partial f}{\partial x_i}(0,\ldots,0) \tag{1.2}
     \end{equation}

     and

     \begin{equation}
        f(x_1,\ldots,x_n)=\sum_{i=1}^{n}{(x_1,\ldots,x_n)}\tag{1.3} 
     \end{equation}

     But since $p_0$ is a critical point, equation $1.2$ turns out to be zero on both 
     sides at $p_0$. So we can apply Theorem $1.1$ again to get $n$ smooth functions 
     $h_{i1},\ldots,h_{in}$ for every $i$ that is defined on $N_{p_0}$ such that

     \begin{equation}
        \sum_{j=1}^{n}{x_jh_{ij}(x_1,\ldots,x_n)}=g_i(x_1,\ldots,x_n) \tag{1.4}
     \end{equation}

     By plugging equation $1.4$ into equation $1.3$, we get

     \begin{equation}
        f(x_1,\ldots,x_n)=\sum_{i=1}^{n}\sum_{j=1}^{n}{x_ix_jh_{ij}(x_1,\ldots,x_n)} \tag{1.5}
     \end{equation}

     We may assume that $h_{ij}=h_{ji}$, rewriting $h_{ij}$ as
     $H_{ij}=\frac{h_{ij}+h_{ji}}{2}$ if necessary. Furthermore,

     \begin{equation}
        (h_{ij}(0,\ldots,0))_{n \times n}=\left( \frac{1}{2}\frac{\partial^2f}{\partial x_i \partial x_j}(0,\ldots,0)_{n\times n} \right) \tag{1.6}
     \end{equation} 
     And since we assumed equation $1.1$ to be true, then $h_{11}(0,\ldots,0)\not = 0$. $h_{11}$
     is a smooth, hence continuous function, and so $h_{11}$ is not zero in a neighborhood of the origin.
     Let us call this neighborhood $\bar{N}_0$    

     Our ultimate goal is to express $f$ in the standard quadratic form of the equation from the lemma.
     We do this by eliminating all terms which are not of the form $\pm x_i^2$ via induction over $k\le n$
     steps. While we are currently dealing with $k=1$, in the general case of $k$, we wish to express $f$ 
     as a sum of terms such that $k$ terms are of the form $\pm x_i^2$ and the rest of the terms depend
     on coordinate in the set ${x_i|i\not = k}$. To this end, let
     \begin{equation}
        G(x_1,\ldots,x_n)=\sqrt{|h_{11}(x_1,\ldots,x_n) | }   \tag{1.6}
     \end{equation}
        
     G is a smooth, non-zero function of $x_1,\ldots,x_n$ on $\bar{N}_0$.

     Now suppose by induction that there exists a local coordinate system $(y_1,\ldots,y_n)$ defined on $\bar{N}_0$ such that
     \begin{equation}
        y_i=x_i(\not = 1) \tag{1.7}
     \end{equation}
     \begin{equation}
        y_1=G*(x_1+\sum_{i>1}^{n}{\frac{x_ih_{1i}}{h_{11}}})\tag{1.8}
     \end{equation}

     It follows from the Inverse Function Theorem that $y_1,\ldots,y_n$ is a local
     coordinate system defined on a smaller neighborhood $\tilde{N}_0 \subset \bar{N}_0$,
     since the determinant of the Jacobian of the transformation from $(x_1, \ldots,x_n)$
     to $(y_1,\ldots,y_m)$ may be verified to be nonzero.

     When we square $y_1$, we get
     \begin{equation}
        y_1^2=\pm h_{11}x_1^2\pm2\sum_{i=2}^{n}{x_1x_ih_{1i}}\pm\frac{\left(\sum_{i=2}^{n}{x_ih_{1i}}\right)^2}{h_{11}}\tag{1.9}
     \end{equation}
     where the signs are either positive or negative, depending on the sign of $h_{11}$. Using equation
     $1.5$, we can verify that $f$ can be expressed in the following way with respect to this
     coordinate system on the restricted domain $\tilde{N}_0$.
     \begin{equation}
        f=\pm y_1^2 + \sum_{i=2}^{n}\sum_{j=2}^{n}{x_ix_jh_{ij}}-\frac{\left(\sum_{i=2}^{n}{x_ih_{1i}}\right)^2}{h_{11}}\tag{1.10}
     \end{equation}
     where the sign of the $y_1^2$ term is positive or negative, depending on the sign of $h_{11}$. Staying
     consistent with our goals, we notice that the first term is in the standard quadratic form seen
     in the Morse Lemma formulation, whereas the rest of the terms depend on local coordinates
     $x_i$ whereby $i\not=k \ (k=1)$. By induction from $k=1$ to $k=n$, we prove the Morse Lemma. $\blacksquare$
% % end N1
%%%%%%%%%%%%%%%%%%%%%%%%%%%%%%%%%%%%%%%%%%%%%%%%%%%%%%%%%%%%%%%%%%%%%%%%%%%%%%%%%%%%%%%%%%%%%%%%%%%%%%%%%%%%%%
%%%%%%%%%%%%%%%%%%%%%%%%%%%%%%%%%%%%%%%%%%%%%%%%%%%%%%%%%%%%%%%%%%%%%%%%%%%%%%%%%%%%%%%%%%%%%%%%%%%%%%%%%%%%%%
% % begin M1
\textbf{Ex.} Show that the function $f: S^2 \to \R, f(x,y,z) = z$ is a Morse function. \\

$f$ is smooth on $ S^2 $ since it extends to a smooth map on all of $\R^3$. We can map $R^3$ to $S^2$ with the stereographic projection two functions:\\
$
	\phi_1(x_1,x_2,x_3) = (\frac{x_1}{1-x_3}, \frac{x_2}{1-x_3} ) 
$
and
$
	\phi_2(y_1,y_2,y_3) = (\frac{y_1}{1+y_3},\frac{y_2}{1+y_3})
$\\
Therefore, to get $S^2 \to R^3$ we can take $\phi_1^{-1}$ and $\phi_2^{-1}$ \\
$
	\phi_1^{-1}(x_1,x_2,x_3) = (\frac{2x_1}{x_1^2+x_2^2+1}, \frac{2x_2}{x_1^2+x_2^2+1}, \frac{x_1^2+x_2^2-1}{x_1^2+x_2^2+1})
$\\
$
	\phi_1^{-1}(y_1,y_2,y_3) = (\frac{2y_1}{y_1^2+y_2^2+1}, \frac{2y_2}{y_1^2+y_2^2+1}, \frac{1-y_1^2+y_2^2}{y_1^2+y_2^2+1})
$\\
Then, we take $g_1 = f \circ \phi_1^{-1}$ and $g_2 = f \circ \phi_2^{-1}$ \\
$
	g_1(x,y) =  \frac{x^2+y^2-1}{x^2+y^2+1}
$
and
$
	g_2(x,y) = \frac{1-x^2+y^2}{x^2+y^2+1}
$ \\
Now, we can compute jacobian of $g_1$ and $g_2$, find critical points, and check that determinant of hessian matrix is non-zero. \\
$
	\nabla g_1(x,y,z) = 0$ iff $(x,y,z) = (0,0,-1)
$\\
$
	\nabla g_2(x,y,z) = 0$ iff $ (x,y,z) = (0,0,1)
$
We have two critical points, and now we compute the hessian at them.\\

$
	H(g_1) = \begin{bmatrix} 4 & 0 \\ 0 & 4 \end{bmatrix}
	\imply \det(H(g_1)) = 16
$ \\

$
	H(g_2) = \begin{bmatrix} -4 & 0 \\ 0 & -4 \end{bmatrix}
	\imply \det(H(g_2)) =  16
$ \\ 

Both critical points are non-degenerate, therefore $f$ is a Morse function.$\blacksquare$\\
% % end M1
%%%%%%%%%%%%%%%%%%%%%%%%%%%%%%%%%%%%%%%%%%%%%%%%%%%%%%%%%%%%%%%%%%%%%%%%%%%%%%%%%%%%%%%%%%%%%%%%%%%%%%%%%%%
%%%%%%%%%%%%%%%%%%%%%%%%%%%%%%%%%%%%%%%%%%%%%%%%%%%%%%%%%%%%%%%%%%%%%%%%%%%%%%%%%%%%%%%%%%%%%%%%%%%%%%%%%%%
% % start V1
\\
\\
Examples of Morse functions: \\
\textbf{Example 1.1}\\
$f(x,y)=e^{xy}+x$\\
First of all we have to find all critical points of $f$. A point $P$ is critical, when $\frac{\partial f}{\partial x}=\frac{\partial f}{\partial y}=0$ at $P$. \\
$\frac{\partial f}{\partial x}= ye^{xy}+1=0 \\
\frac{\partial f}{\partial y}=xe^{xy}=0$
After solving the system of equations we get that $x=0$, $y=-1$. Hence, the only critical point of $f$ is $(0,-1)$. Now to prove that the function is a Morse function we have to shoe that the determinant of Hessian matrix at the point $(0,-1)$ is non-zero. \\
Partial second order derivatives are: \\
$\frac{\partial^2 f}{\partial x^2}=y^2e^{xy}$
$\frac{\partial^2 f}{\partial x^2}(0,-1)=1$\\
$\frac{\partial^2 f}{\partial \partial y}=e^{xy}+yxe^{xy}$
$\frac{\partial^2 f}{\partial \partial y}(0,-1)=1$\\
$\frac{\partial^2 f}{\partial y \partial x}=e^{xy}+xye^{xy}$
$\frac{\partial^2 f}{\partial y \partial x}(0,-1)=1$\\
$\frac{\partial^2 f}{\partial y^2}=x^2e^{xy}$
$\frac{\partial^2 f}{\partial y^2}(0,-1)=0$\\
Therefore, \\
$det(Hess(f(0,-1)))=\left| \begin{array}{cc} 1 & 1 \\ 1 & 0 \end{array} \right|=-1$. \\
Since the determinant is non-zero at the only critical point of the function, we can claim that it is a Morse function. \\
\textbf{Example 1.2}\\
Let's consider a function $g(x,y)=xy$ and prove that it is , indeed, a Morse function.\\
First order partial derivatives: \\
$\frac{\partial g}{\partial x}=y$\\
$\frac{\partial g}{\partial y}=x$\\
Therefore, the only critical point of $g$ is $(0,0)$. Now we have to evaluate the determinant of Hessian matrix at this point.\\
Second order partial derivatives: \\
$\frac{\partial^2 g}{\partial x^2}=0$\\
$\frac{\partial^2 g}{\partial x \partial y}=1$\\
$\frac{\partial^2 g}{\partial y \partial x}=0$\\
$\frac{\partial^2 g}{\partial^2 y}=1$\\
Hence the determinant of Hessian at the critical point is: \\
$det(Hess(f(0,0)))=\left| \begin{array}{cc} 0 & 1 \\ 1 & 0 \end{array} \right|=-1$. \\
We observe that the determinant is nonzero, hence, the only critical point of $g(x,y)=xy$ is non-degenerate, so the function is Morse. \\
\\
\\
Examples of functions that are not Morse: \\
\textbf{Example 2.1}\\
$f(x,y)=x^3+xY^2-x^2y-y^3$\\
Let's follow the previous procedure: \\
$\frac{\partial f}{\partial x}=3x^2+y^2-2xy=0$\\
$\frac{\partial f}{\partial y}=2xy-x^2-3y^2=0$\\
After solving the system of equations we got that the only critical point of $f$ is $(0,0)$. Now compute second order partial derivatives:\\
$\frac{\partial^2 f}{\partial x^2}=6x-2y$\\
$\frac{\partial^2 f}{\partial x \partial y}=2y-2x$\\
$\frac{\partial^2 f}{\partial y \partial x}=2y-2x$\\
$\frac{\partial^2 f}{\partial^2 y}=2x-6y$\\
Therefore, \\
$det(Hess(f(0,0)))=\left| \begin{array}{cc} 0 & 0 \\ 0 & 0 \end{array} \right|=0$.\\
Hence, the only critical point of $f$ is degenerate, so $f$ is not a Morse function. \\
\textbf{Example 2.2}\\
$g(x,y)=x^3$\\
$\frac{\partial g}{\partial x}=3x^2$\\
$\frac{\partial g}{\partial y}=0$\\
Therefore, critical points are all points of the form $(0, y_1)$.\\
$\frac{\partial^2 g}{\partial x^2}=6x$
$\frac{\partial^2 g}{\partial x \partial y}=0$\\
$\frac{\partial^2 g}{\partial y \partial x}=0$
$\frac{\partial^2 g}{\partial^2 y}=0$\\
Hence, Hessian will be of the form: \\
$Hess(g(0,y_1))=\left| \begin{array}{cc} 0 & 0 \\ 0 & 0 \end{array} \right|,  \forall y_1$. \\
Therefore, all critical points of $g$ are degenerate points, moreover, values of $g$ at all critical points are equal, hence, $g(x,y)=x^3$ is not a Morse function. 
% % end V1
%%%%%%%%%%%%%%%%%%%%%%%%%%%%%%%%%%%%%%%%%%%%%%%%%%%%%%%%%%%%%%%%%%%%%%%%%%%%%%%%%%%%%%%%%%%%%%%%%%%%%%%%%%%%%%%%%%%%%%%%%%%%%%%%%%%%%%%%%%%%%%%%%%%%%%%%%%%%%%%%%%%
%%%%%%%%%%%%%%%%%%%%%%%%%%%%%%%%%%%%%%%%%%%%%%%%%%%%%%%%%%%%%%%%%%%%%%%%%%%%%%%%%%%%%%%%%%%%%%%%%%%%%%%%%%%%%%%%%%%%%%%%%%%%%%%%%%%%%%%%%%%%%%%%%%%%%%%%%%%%%%%%%%
% % begin N2 
% % end N2
%%%%%%%%%%%%%%%%%%%%%%%%%%%%%%%%%%%%%%%%%%%%%%%%%%%%%%%%%%%%%%%%%%%%%%%%%%%%%%%%%%%%%%%%%%%%%%%%%%%%%%%%%%%%%%%%%%%%%%%%%%%%%%%%%%%%%%%%%%%%%%%%%%%%%%%%%%%%%%%%%%
%%%%%%%%%%%%%%%%%%%%%%%%%%%%%%%%%%%%%%%%%%%%%%%%%%%%%%%%%%%%%%%%%%%%%%%%%%%%%%%%%%%%%%%%%%%%%%%%%%%%%%%%%%%%%%%%%%%%%%%%%%%%%%%%%%%%%%%%%%%%%%%%%%%%%%%%%%%%%%%%%%



% % begin M2
Now, we explore Reeb graphs on manifolds of dimension 2, such that functions acting on them is a Morse function. In this case, we have a bijection between the critical points of $f$ and the nodes of $R(f)$ . We introduce the following definitions, to be able to see interesting properties of reeb graphs on orientable 2-manifolds
\begin{definition}
    Loop is a continuous function $X \to I$ where $X$ is some topological space, and $I=[0,1]$, s.t. $f(0) = f(1)$
\end{definition}
\begin{definition}
    A surface is orientable if and only if any loop on it is orientable preserving.
\end{definition}
\begin{definition}
    Loop is orientation-preserving if an only if there is a continuous choice of tangent frame along the loop such that the frame at the beggining is the same as the frame at the end
\end{definition}
For example, mobius strip is a non-orientable manifold, since if we take a normal vector accross the following loop, after one full circle it ends up in the same point but pointint to different direction.
\begin{center}
\includegraphics[width=.4\textwidth]{m1.png}
\includegraphics[width=.4\textwidth]{m2.png}
\includegraphics[width=.4\textwidth]{m3.png}
\includegraphics[width=.4\textwidth]{m4.png}
\end{center}

\textbf{Orientable Manifolds} \\
Our goal is to express number of loops in a Reeb graph by # of critical points. Let $n_i$ be the number of nodes with degree $i$. Orientable manifolds have 0 $n_2$ because every point is either minimum, maximum or saddle.\\
Therefore the number of arcs is number of connections each point has, divided by two (because we count each arc twice) is $e = (n_1+3n_3)/2$ \\ 
number of loops is $1+e-(n_1+n_3)$
\begin{definition}
    Let P be a polyhedron with V vertices, E edges, and F faces. Then we define the Euler characteristic to be 
    \begin{equation}
        \chi(P) = V - E + F  
    \end{equation}
\end{definition} \\
More generally, Euler characteristic is defined as alternating sum of Betti numbers. Last Morse inequality allow us to define $\chi$ as a set of inequalities for alternating sums of Betti numbers in terms of a corresponding alternating sum of the number of critical point $n_i$ of a Morse function of a given index.
\begin{lemma}
    The Reeb graph of a Morse function on a connected, orientable 2-manifold of genus g has g loops.
    \begin{proof}
        We first suppose that Reeb graph has no loops. It has $n_1 = n_3+2$. And if we write $c_i$ for the number of critical points of index $i$ we have $n_i = c_0+c_2$ and $n_3 = c_1$. we have $\chi = c_1 - c_1 +c_2 = n_1 - n_3 = 2.$  2 is the Euler characteristic of a sphere, which has no genus. \\ 
        Now suppose that Reeb graph has at least one loop. We can apply following two functions because they preserve homotopy type and the number of loops: collapse degree 1 nodes and merge arcs across degree 2 nodes which get eliminated in the process. For example consider the following graph:
    \begin{center}
    \includegraphics[width=0.4\textwidth]{homotopic_transformation.png}
    \end{center}
    Let $m_3$ be the number of remaining degree 3 nodes and note that it is even because $3m_3$ is twice the number of remaining arcs. Using the Euler-Poincare Theorem for graphs we get $\chi = m_3 - 3m_3/2 = \beta_0 - \beta_1$. The graph is connected which implies that the number of loops is $\beta = m_3/2+1$. We have $c_1$ degree 3 nodes in the original Reeb graph and for each minimum and maximum we collapse one degree 1 node removing a 3 degree node in the process. Last strong Morse inequality tells us that $m_3 = c_1 - (c_0 + c_2) = -\chi = 2g - 2$. The number of loops is therefore $ \beta_1 = (2g-2)/2+1=g$
    \end{proof}
\end{lemma}
% % end M2
%%%%%%%%%%%%%%%%%%%%%%%%%%%%%%%%%%%%%%%%%%%%%%%%%%%%%%%%%%%%%%%%%%%%%%%%%%%%%%%%%%%%%%%%%%%%%%%%%%%%%%%%%%%%%%%%%%%%%%%%%%%%%%%%%%%%%%%%%%%%%%%%%%%%%%%%%%%%%%%%%%
%%%%%%%%%%%%%%%%%%%%%%%%%%%%%%%%%%%%%%%%%%%%%%%%%%%%%%%%%%%%%%%%%%%%%%%%%%%%%%%%%%%%%%%%%%%%%%%%%%%%%%%%%%%%%%%%%%%%%%%%%%%%%%%%%%%%%%%%%%%%%%%%%%%%%%%%%%%%%%%%%%
% % begin V2
% % end V2
%%%%%%%%%%%%%%%%%%%%%%%%%%%%%%%%%%%%%%%%%%%%%%%%%%%%%%%%%%%%%%%%%%%%%%%%%%%%%%%%%%%%%%%%%%%%%%%%%%%%%%%%%%%%%%%%%%%%%%%%%%%%%%%%%%%%%%%%%%%%%%%%%%%%%%%%%%%%%%%%%%
\begin{thebibliography}{9}
     \bibitem{s}
         {\sc }
         `` https://core.ac.uk/download/pdf/20670562.pdf
\end{thebibliography}

\caption{ 4 figures}

\textbf{Non-orientable 2-manifolds without boundary}\\

We will consider in this section Reeb graphs of Morse functions over non-orientable 2-manifolds with boundary. In particular, number of nodes, contours and loops in such graphs. \\
All connected non-orientable 2-manifolds without boundary could be expressed as connected sum of  $g$ copies of projective planes $N=P\# P...\# P$, where $g$ coincides with the genus of the surface.\\
For the further proofs we have to recall that Euler characteristic of orientable surfaces can be calculated as $\chi=2-2g$ and for non-orientable it is $\chi=2-g$. Also we have to notice that we can make a 2-sheeted cover of a non-orientable 2-manifold, where each sheet is orientable. In future we will call this $doubling$. \\
Let's consider the process of doubling. We obtain 2-sheeted cover of $N$ (non-orientable 2-manifold) by doubling each point $x\in N$ to two points $x'$ and $x''$. We can consider them as a points lying in the neighborhood of $x$ on two opposite sides of $N$. Union of such points $x'$ and $x''$ will be space that is connected and orientable 2-manifold $M$. Due to the doubling Euler characteristic of the surface will be doubled, $\chi_M=2\chi_N$. \\
Now we have to calculate genus of $M$. Since it is orientable:\\ $g_M=\frac{2-\chi_M}{2}=1-\frac{\chi_M}{2}=1-\chi_N=1-(2-g_N-1)=g_N-1$. \\
Therefore, by doubling we decrease genus by 1. \\
In the table on the fig. \ref{fig:Table} one can find Euler characteristicand genus number of basic non-orianatble manifolds. Note that $K$ is a Klein bottle, $T$ represents torrus,$S$ sphere and $P$ is a projective plane. 
\begin{figure}[h!]
\center{\includegraphics[scale=0.9]{char_table.png}}
\caption{Euler characteristic and genus of some non-orientable manifolds and their doublings}
\label{fig:Table}
\end{figure}
On the table fig. \ref{fig:Table} one can observe that relation of Euler characteristic and genus number of non-orientable manifolds and their doublings coincides with the dependence introduced above. \\
Now we can move on to describing Reeb graphs of non-orientable 2-manifolds without boundary. As an example let's consider Klein bottle, its Reeb graph, and Reeb graph of its doubling surface on fig.\ref{fig:Klein}. \\
\begin{figure}[h!]
\center{\includegraphics[scale=0.45]{Klein_bottle.png}}
\caption{Reeb graph of Klein bottle and its doubling}
\label{fig:Klein}
\end{figure}
As we see on the fig. \ref{fig:Klein} Reeb graph of the surface drastically changes after doubling, especially number of nodes and contours.  Let's learn how do nodes of each degree change under this doubling. \\
1-degree nodes are maxima or minima, after doubling number of these nodes is doubled and also number of contours is doubled as well. \\
2-degree nodes exist in Reeb graphs of non-orientable surfaces, despite the fact that in orientable surfaces they could be just collapsed. It is transformed into 4-degree node after doubling. The reason is that the level set containing a critical point of index 1 is a figure-8 which may contain an orientation-reversing cycle. \\
3-degree nodes are saddle points that just transfer into two such nodes in its doubling. \\
One could observe the above on the fig. \ref{fig:Nodes}.  
\begin{figure}[h!]
\center{\includegraphics[scale=0.3]{Nodes.png}}
\caption{Correspondence between nodes in $N$ and its doubling $M$}
\label{fig:Nodes}
\end{figure}
Let $f: N \rightarrow R$ is a Morse function that induces function $f_0: M \rightarrow R$. $f_0$ is not a Morse function, since all critical points of $f$ correspond to two critical points of $f_0$, so they are not distinct. However, critical points of $f_0$ are still non-degenerate, that gives as an opportunity to consider $R(f_0)$ (Reeb graph of $f_0$) as previously. \\
Let $e$ be number of arcs in $R(f)$, $n_{1,3}$ is number of 1-degree and 3-degree nodes of $R(f)$, and $n_2$ is number 2-degree nodes. As seen previously, number of 1,3-degree nodes and their contours are doubled in $M$. However, number of 2-degree nodes is invariant under doubling, corresponding contours are doubled. \\
Therefore, $R(f_0)$ has $2e$ arcs and $2n_{1,3}+n_2$ nodes. Known fact that number of loops in a graph is $\beta_1 (G)=1-+e-n$, where $e$ is number of edges(arcs) and $n$ is number of nodes. \\
After substitution we get:\\
$1-\beta_1(R(f_0)=2n_{1,2}+n_2-2e=2-2+2n_{1,3}+2n_2+2e-n_2=2-2(1-n_{1,3}-n_2+e)-n_2=2-2\beta_1(R(f))-n_2$. \\
Also we know that number of loops of $N$ is its genus and genus of $M$ (number of loops) is one less that one of $N$. Therefore, we can make a substitution $\beta_1(R(f_0))=g_N-1$. \\
From the above results we get: \\
$\beta_1(R(f))=\frac{g_N-n_2}{2}$.\\
Since number of 2-degree nodes $n_2\geq0 $, we may obtain the following result:\\
\textbf{Lamma ?}\\
The Reeb graph of a Morse function over connected non-orientable 2-manifolds of genus $g$ with boundary is at most $\frac{g}{2}$.\\

\end{document}
